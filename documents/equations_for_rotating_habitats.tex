\documentclass[a4paper]{article}
\usepackage[margin=3cm]{geometry}

\usepackage{hyperref}
\hypersetup{
	colorlinks=true,
	linkcolor=blue,
	filecolor=magenta,
	urlcolor=cyan,
	hyperindex=true,
	linktocpage=true,%makes the page number be the link in the index
}

\title{Equations for rotating habitats}

\author{Chaya2766}

\begin{document}
	\sffamily
	\sloppy
	
	\maketitle
	
	\vfill
	
	\begin{abstract}
		Equations for centrifugal force are provided in multiple forms, including those which rely on time taken to complete a full rotation instead of angular velocity.
		
		\medskip
		
		An equation for maximum radius of a rotating habitat is also provided, as well as an explanation of how it was constructed from centrifugal force and hoop stress equations.
		
		\medskip
		
		Equations for sideways acceleration experienced by people climbing staircases or riding elevators is given, based on my own derivation of the coriolis force, which is notably less general than the commonly given equations for it and in disagreement with them by a factor of 2, but is explained in terms of the change in tangential velocity which is not visible in the rotating frame of reference.
		
		\medskip
		
		A way to calculate the mass of a rotating habitat is also provided, with considerations for radiation shielding and ability to add buildings inside the habitat.
		
		\medskip
		
		In all 4 cases, one or more examples for results of the calculations are provided as reference.
	\end{abstract}
	
	%\noindent \rule{\linewidth}{1pt}
	\vfill
	
	\tableofcontents
	
	\pagebreak
	
	\section{Centrifugal gravity}
	
	Due to my observation that people tend to end up wrong by a factor of pi or 2 pi when running calculations for broadly anything rotation-dependent (including myself the first few times I tried to tackle rotating habitats), I had converted the formulas into additional forms which rely on time to complete one full rotation instead of angular velocity.
	
	\begin{center}
		\textbf{equations for centrifugal gravity}
	\end{center}
	
	$$ t = 2\pi\omega = \frac{2 \pi}{\sqrt{\frac{a}{r}}} $$
	
	$$ \omega = \frac{2 \pi}{t} = \sqrt{\frac{a}{r}} $$
	
	$$ r =  \frac{a}{\omega^2}= a \cdot \left( \frac{t}{2 \pi} \right)^2 = \frac{v^2}{a}$$
	
	$$ a = \frac{r}{\omega^2} = \frac{r}{\left( \frac{t}{2 \pi} \right)^2} = \frac{v^2}{r}$$
	
	$$ v = \omega \cdot r = r \cdot \frac{2 \pi}{t} = \sqrt{a \cdot r}$$
	
	where:
	
	\begin{itemize}
		\item $t$ is time to complete a full rotation
		
		\item $a$ is the centrifugal acceleration felt on the inside
		
		\item $r$ is the radius of the station
		
		\item $\omega$ is the angular velocity
		
		\item $v$ is tangential velocity
	\end{itemize}
	
	\medskip
	
	\noindent \rule{\linewidth}{1pt}
	
	\medskip
	
	For a simple example, a habitat 200m in radius, which completes one rotation every 28.099 seconds would have an internal gravity (a) of 10m/s$^2$ at the floor. It would also have an angular velocity of 0.2236 rad/s and tangential velocity of 44.72m/s at the floor.
	
	\pagebreak
	
	\section{Sideways coriolis force}
	
	Be warned that this is my own derivation of the effect, and it notably does not match the formulas for "coriolis force" shown elsewhere. My attempts to research the coriolis force so far failed to let me make sense of the factor of two commonly included in the formula for it. My interpretation of it is also evidently different, as I focussed on accelerations that are parallel to the floor, instead of generalizing it for any direction. For this reason I had opted to title the section as "sideways coriolis force" instead of simply "coriolis force" as to not lead others into a mistake. I still believe that the effect I am describing makes physical sense, as explained after the equations, even if it is different from the general "coriolis~force".
	
	\medskip
	
	\begin{center}
		\textbf{sideways acceleration caused by coriolis force}
	\end{center}
	
	$$ a_c = \omega \cdot v_v = \frac{2 \pi}{t} \cdot v_v $$
	
	$$ \Delta v_h = \omega \cdot \Delta r = \frac{2 \pi}{t} \cdot \Delta r $$
	
	where:
	
	\begin{itemize}
		\item $a_c$ is the acceleration caused by the coriolis force
		
		\item $v_v$ is vertical component of the object's velocity\\(meaning velocity with which the object is moving towards the axis of rotation)
		
		\item $t$ is time in which the habitat completes one full rotation
		
		\item $\omega$ is the angular velocity
		
		\item $\Delta v_t$ is the change in an object's tangential velocity \underline{as measured in the rotating frame of reference}\\an object sitting on the floor in any part of the habitat is considered to have a velocity of 0m/s in this understanding
		
		\item $\Delta r$ is the change in an object's distance from the axis of rotation
	\end{itemize}
	
	\medskip
	
	\noindent\rule{\linewidth}{1pt}
	
	\medskip
	
	For the numbers in the following example I will be using a habitat 200m in radius with a full rotation time of 28.099s, the same as in the centrifugal force example.
	
	\medskip
	
	The effect can be conceptualized like this: think of standing at the bottom of a staircase 200m away from the axis of rotation - you consider yourself stationary even though you know that in reality you and the floor you are standing on are both moving around the station's axis of rotation at 44.72m/s.
	
	\medskip
	
	When you take the first step onto the staircase, your foot will be stationary on the first step, which is 20cm higher, placing it at 199.8m away from the axis of rotation, meaning that its tangential velocity is 44.677m/s, which is less than the floor you were standing on. As such your foot must have accelerated by 0.0447m/s as you raised it to match the new tangential velocity.
	
	$$ \Delta v_h = \frac{2pi}{28.099s} \cdot 0.2m = 0.04472m/s $$
	
	You would have felt this as a force accelerating your foot in the direction of the station's spin, which you would have to counteract. Since tangential velocity is proportional to distance from the axis of rotation, you will have to counteract the same velocity change for every next step.
	
	\medskip
	
	This same effect would cause you to experience a sideways pull if you were in an elevator, in an example if you are moving upwards at 10m/s then you would be feeling an apparent sideways acceleration of 2.236m/s$^2$
	
	$$ a_c = \frac{2pi}{28.099s} \cdot 10m/s = 2.236m/s^2 $$
	
	\pagebreak
	
	\iffalse
	
	\section{Coriolis force}
	
	Coriolis force is another strange force that has to be accounted for in some cases, such as when deciding the direction of staircases and speeds of elevators.
	
	\begin{center}
		\textbf{strength of acceleration caused by coriolis force}
	\end{center}
	
	$$ a_c = 2 \cdot v_r \cdot \omega $$
	
	where:
	
	\begin{itemize}
		\item $t$ is time to complete a full rotation
		
		\item $a$ is the acceleration caused by the coriolis force
		
		\item $\omega$ is the angular velocity
		
		\item $v_r$ is the velocity of the object as \underline{measured in the rotating frame of reference}
		
		(in simple terms, velocity relative to a point on the floor)
	\end{itemize}
	
	\medskip
	
	\noindent\rule{\linewidth}{1pt}
	
	\medskip
	
	Understanding where the "2" in that equation comes from had been a problem for me for the longest time, so I will provide here the thought experiment I constructed which clarified it to me.
	
	\medskip
	
	Consider the example habitat described at the bottom of the previous section. A person standing on the floor has a tangential velocity of 44.72m/s in the non-rotating view, but in their own view they have a velocity of 0m/s.
	
	\medskip
	
	Now consider that they throw a ball exactly in the oposite of the station's spin direction, at exactly 44.72m/s. In that case the ball will have 0m/s of velocity in the non-rotating view, and as such should experience no centrifugal acceleration pulling it downwards to the floor.
	
	\medskip
	
	The coriolis force equation cares for the velocity as measured in the rotating frame of reference, so in the rotating frame of reference the person is stationary and the ball is moving at 44.72m/s. The equation provided says the ball should be experiencing 20m/s$^2$ of acceleration due to coriolis force (and due to the direction of its movement, that acceleration would be directed upwards).
	
	\medskip
	
	$$ 2 \cdot 44.72m/s \cdot 0.2236rad/s = 19.99878m/s \approx 20m/s $$
	
	\medskip
	
	Looking at the problem from the view of the rotating frame of reference only, you can think of the centrifugal acceleration as constant regardless of object motion, and in that case it would be the job of the coriolis force to perfectly cancel it out for an object which is moving against the spin like that. Take note of the fact the station is curved, and if the acceleration were perfectly cancelled out the ball would still hit the floor, so as seen by the person inside the station it must actually be constantly accelerating upwards to avoid hitting the floor - this explains why the equation needs the factor of 2. This is ofcourse an illusion, which comes clear when you consider the situation from the non-rotating point of view.
	
	\pagebreak
	
	This approach of setting the centrifugal acceleration as constant and using the coriolis force instead also works for running people, as it returns the same values as simply changing the tangential velocity in the centrifugal acceleration formula. See in example for the person running in the direction of the spin at 15m/s:
	
	$$ a_{stationary} = \frac{(44.72m/s)^2}{200m} = 10m/s^2 $$
	
	$$ a_{coriolis} = 2 \cdot 15m/s \cdot 0.2236rad/s = 6.708m/s^2 $$
	
	$$ a_{running} = \frac{(44.72m/s + 15m/s)^2}{200m} = 17.832m/s^2 $$
	
	$$ a_{stationary} + a_{coriolis} = 10m/s^2 + 6.708m/s^2 = 16.708m/s^2 \neq 17.832m/s^2 $$
	
	I had briefly achieved enlightenment and then taken it away from myself :(
	
	\fi
	
	\pagebreak
	
	\section{Maximum habitat radius}
	
	The final equation is presented here first and the reasoning behind it is on the next page.
	
	\begin{center}
		\textbf{Maximum radius of a rotating habitat}
	\end{center}
	
	$$ R_{max} = \frac{\sigma_{max}}{s \cdot g \cdot \rho} $$
	
	$$ s = \frac{\sigma_{max}}{R_{max} \cdot g \cdot \rho} $$
	
	where:
	
	\begin{itemize}
		\item $\sigma_{max}$ is the maximum tensile stress the building material can handle
		
		\item $R_{max}$ is the largest achievable radius of the cylinder
		
		\item $s$ is the safety factor (eg. 2 means the actual stress on the habitat wall would have to double for it to break, you can interpret it to mean how many times stronger the wall material is than necessary)
		
		\item $g$ is the surface gravity felt in the habitat
		
		\item $\rho$ is the density of the material from which the habitat is made
	\end{itemize}
	
	Safety factor must always be larger than 1, if the safety factor is exactly 1 then the habitat is operating exactly at its theoretical breaking point, where the centrifugal force acting on the mass of its walls is just enough to overpower the tensile strength of the walls.
	
	\bigskip
	
	\noindent \rule{\linewidth}{1pt}
	
	\bigskip
	
	Example of a rotating habitat with safety factor 2 and internal gravity 1g, made from mild steel of density $\rho = 7850kg/m^3$ and tensile strength $\sigma = 300MPa$:
	
	$$ R_{max} = \frac{300MPa}{2 \cdot 9.807m/s^2 \cdot 7850kg/m^3} = 1948m $$
	
	This means the habitat as a whole could be 3.896km in diameter, potentially a bit less than 8km in diameter if a lower safety factor were acceptable.
	
	\bigskip
	
	Several other materials using the same assumptions of 1g internal gravity and safety factor of 2:
	
	\begin{enumerate}
		\item Pure aluminium ($\rho = 2699kg/m^3$, $\sigma = 47MPa$): 887.8m
		
		\item fully annealed \href{https://en.wikipedia.org/wiki/SAE_304_stainless_steel#Mechanical_properties}{SAE 304 stainless steel} ($\rho = 7900kg/m^3$, $\sigma = 210MPa$): 1355m
		
		\item Pure titanium ($\rho = 4110kg/m^3$, $\sigma = 434MPa$): 5384m
		
		\item fully hardened \href{https://en.wikipedia.org/wiki/SAE_304_stainless_steel#Mechanical_properties}{SAE 304 stainless steel} ($\rho = 7900kg/m^3$, $\sigma = 1050MPa$): 6776m
		
		\item \href{https://en.wikipedia.org/wiki/7034_aluminium_alloy}{7034 aluminium alloy} ($\rho = 2890kg/m^3$, $\sigma = 730MPa$): 12.88km
		
		\item Carbon fiber ($\rho = 1800kg/m^3$, $\sigma = 4GPa$): 113.3km
		
		\item Low end carbon nanotubes ($\rho = 1400kg/m^3$, $\sigma = 11GPa$): 400.6km
		
		\item High end carbon nanotubes ($\rho = 1400kg/m^3$, $\sigma = 63GPa$): 2294km
	\end{enumerate}
	
	\pagebreak
	
	The equation comes from combining hoop stress with centrifugal gravity. Hoop stress is the stress that appears in the walls of a container as a result of its presurized contents pushing out on the walls from the inside. For a cylindrical container it is usually formulated as this:
	
	$$ \sigma_\theta \approx \frac{R_c \cdot P}{t_d} $$
	
	where:
	
	\begin{itemize}
		\item $\sigma_\theta$ is the hoop stress in the cylinder wall
		
		\item $R_c$ is the radius of the cylinder
		
		\item $P$ is the pressure pushing out on the cylinder wall
		
		\item $t_d$ is the thickness of the cylinder wall
	\end{itemize}
	
	The approximation is due to "thin-walled" assumption, that the thickness of the wall is insignificant compared to the size of the whole container, usually specified as the thickness being no more than 1/20 of the whole container's diameter. This is nearly guaranteed to be the case for O'Neill cylinders and similar rotating habitats.
	
	\medskip
	
	For a rotating habitat, the centrifugal gravity will be the "pressure" pushing out on it from the inside. Take note of the similarity, that for any given internal surface along the wall, a cylindrical container with pressurized contents is experiencing a force pushing out on it directed outwards from the central axis of the container. In a rotating habitat, the centrifugal acceleration also points away from the central axis, which creates a similar force due to the mass of the walls of the habitat. This force can be divided by the total area of the walls to calculate the equivalent pressure:
	
	$$ P \approx \frac{g \cdot m_{wall}}{a_{wall}} $$
	
	where:
	
	\begin{itemize}
		\item $P$ is the pressure pushing out on the cylinder wall
		
		\item $g$ is the surface gravity in the cylinder
		
		\item $m_{wall}$ is the total mass of the cylinder wall
		
		\item $a_{wall}$ is the surface area of the cylinder wall
	\end{itemize}
	
	Plugging in this "pressure" into the hoop stress equation transforms it into a new form:
	
	$$ \sigma_\theta \approx \frac{R_c \cdot g \cdot m_{wall}}{t_d \cdot a_{wall}} $$
	
	The mass of the habitat is a function of the wall thickness and density of the material it is made from, and can also be approximately related to the floor area:
	
	$$ m_{wall} \approx t_d \cdot a_{wall} \cdot \rho_{wall}$$
	
	where:
	
	\begin{itemize}
		\item $\rho_{wall}$ is the density of the wall material
		
		\item $t_d$ is the thickness of the cylinder wall
		
		\item $m_{wall}$ is the mass of the cylinder wall
		
		\item $a_{floor}$ is the surface area of the habitat floor
	\end{itemize}
	
	Plugging in this equation for the mass of the wall into our equation cancels out the area and thickness of the wall further transforms it into the finally useable form:
	
	$$ \sigma_\theta \approx R_c \cdot g \cdot \rho_{wall} $$
	
	Formulated like this, it gives us the stress experienced by the habitat wall as a function of the habitat's radius, internal gravity and density of the material the wall is made from, and can be used to judge what material is required for given habitat specifications.
	
	\pagebreak
	
	\section{Habitat mass}
	
	Calculating a habitat's mass is very straightforward:
	
	$$ M = \rho_a \cdot (2\pi \cdot R_c \cdot L + 2 \pi \cdot R_c^2) $$
	
	where:
	
	\begin{itemize}
		\item $M$ is the total mass of the habitat
		
		\item $\rho_a$ is the aerial density of the habitat wall (mass per area, typically set to 10t/m$^2$)
		
		\item $R_c$ is the radius of the habitat
		
		\item $L$ is the length of the habitat
	\end{itemize}
	
	It is really just multiplying the aerial density by the surface area of a cylinder. That equation also assumes flat end caps, so if one uses hemispherical end caps instead (which is generally preferable) there would be one slight change:
	
	$$ M = \rho_a \cdot (2\pi \cdot R_c \cdot L + 4 \pi \cdot R_c^2) $$
	
	\medskip
	
	Taking as example the classic \href{https://en.wikipedia.org/wiki/O%27Neill_cylinder#Islands}{Island Three} design, 8km diameter, 32.2km length plus hemispherical caps, and 10t/m$^2$ aerial density:
	
	$$ M = 10t/m^2 \cdot (2\pi \cdot 4km \cdot 32.2km + 4\pi \cdot (4km)^2) =  1.010336e13kg $$
	
	\noindent \rule{\linewidth}{1pt}
	
	\medskip
	
	Considerations for deciding a habitat's aerial density:
	
	\begin{enumerate}
		\item Radiation shielding - a commonly quoted figure is 10 tonnes per square metre to achieve good shielding. This commonly quoted figure most likely comes from the fact that \href{https://en.wikipedia.org/wiki/Atmosphere_of_Earth}{earth's atmosphere} masses 5.15e18kg, and when divided over earth's 5.1e14m$^2$ of surface area gives almost exactly that aerial density.
		
		\medskip
		
		If radiation shielding is not quite as needed for whatever reason, the aerial density could be lowered to proportionally decrease the habitat's mass.
		
		\item Support of internal structures - this depends on the "safety factor" mentioned earlier in the maximum habitat radius section. If a habitat fulfills the quoted 10 tonnes per square metre figure and is built such that it's own centrifugal gravity stresses the walls to only half their yield strength, then every square metre of the wall will be able to support at most 10 tonnes of additional buildings on top of it.
		
		\medskip
		
		Generally any reasonably small section of habitat wall will be able to support s-1 times it's own mass worth of buildings and other structure built on top of it inside the habitat (s is the safety factor the habitat was built with) so the more mass per area you give the wall, the more you can build inside the habitat.
	\end{enumerate}
	
	\medskip
	
	In any case it should be notable that earth itself, while providing 10 tonnes per square metre of radiation shielding, does not actually provide one square metre of living area per 10 tonnes of mass. Most of \href{https://en.wikipedia.org/wiki/Earth_mass}{earth's 5.9722e24kg of mass} is practically useless, and would give it an aerial density of 11~708~543.11t/m$^2$ if one counts the whole surface area, or worse yet 40~098~026.05t/m$^2$ if one only counts the land area.
	
	\medskip
	
	This means that at baseline any rotating habitat adhering to the 10 tonnes per square metre rule will be a roughly 11 to 40 million times better than an earthlike planet at providing the most living area with the available mass, hinting at what you could do if you took apart the planets in a star system for their building material.
	
\end{document}